%! Author = adnansiddiquei
%! Date = 20/12/2023

\subsection{Q4 - Baseline Dataset}\label{subsec:q4}
\subsubsection{Question 4a}\label{subsubsec:q4a}
    Classification decision trees utilise recursive binary splitting to split the feature-space into high dimensional
    rectangles in order to predict the outcome class given a feature set.
    In each iteration, an existing rectangle is split into two new rectangles, by splitting on a feature and a threshold
    value for that feature.
    The end tree will have some number of internal nodes $N$, representing splits in the feature space, and $N + 1$
    terminal nodes which represent the final classification, which would be the modal class in that rectangle.
    The quality of the split can be measured by a criterion such as the Gini index, and at each split, the feature to
    split on and the threshold would be chosen by whichever split decreases the Gini index the most.
    The Gini index is defined as \cite{ISL}:
    \begin{equation}
        G = \sum_{k=1}^{K} \hat{p}_{mk}(1 - \hat{p}_{mk})
        \label{eq:gini-index}
    \end{equation}
    where $\hat{p}_{mk}$ is the proportion of training samples in the $m$th region that are from the $k$th class.
    A region $m$ that contains mainly a single class will have a small Gini index for that region.

    Decision trees suffer from high variance as the trained model is highly dependent on the training data.
    Bagging and random forests are ensemble methods that reduce the variance of decision trees.
    With bagging, multiple decision trees are trained on different bootstrapped samples of the training data, and the
    classification is then decided by a majority vote of the trees.
    Random forests reduce the variance further by forcing an internal node to split on a random subset of the features,
    resulting in less correlated trees, and therefore a larger decrease in variance.

    TODO THE LAST PART OF THIS QUESTION

\subsubsection{Question 4b}\label{subsubsec:q4b}
